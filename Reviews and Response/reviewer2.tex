\documentclass[]{letter}
\begin{document}
\textbf{Reviewer 2}

The paper titled ``The Correspondence Analysis of Two-Mode Networks Revisited" deals with a classic formal argument regarding centrality in two-mode networks, i.e. that of using information on the other mode to compute centrality for a given mode of the network. It proposes a revision of Bonacich's proposal for calculating two-mode centrality that deploys Correspondence Analysis (CA) to obtain dual centrality scores in a ``reflective" fashion. In addition, it provides an attempt to revise the use of CA for analysing the relational patterns of two-mode networks.

The paper is interesting and full of accurate formal reasoning around the above topics, recalling back Bonacich's original argument and comparing it with a version of reflective centrality, then furtherly developing the discussion by taking into account the ``generalized relational similarity" (GRS) approach to two-mode networks.

However, although the paper does much effort to illustrate the potentiality of CA to better understand the relational patterns of two-mode networks, with a rich and informative suite of commentary and graphical representations, I've got several reservations to make about this work - mainly issues of argument strength, clarity, and lack of reference to other relevant topics - and I shall describe them as follows.

The paper mentions the important legacy with Breiger's seminal work on duality in affiliation (two-mode) networks, as it provides a key grounding to the elaboration of centrality in such networks, stressing the bond between the two modes in analysing them. In my opinion, though, the paper does not take Breiger's argument in due consideration, even more as the piece is aimed at contributing to a special issue on ``Duality in social networks". The latter principle seems to remain in the background, while it should explicitly inform most of the analyses presented in this work.

I suggest extending the introduction, enriching the discussion of the background as for CA in the context of two-mode networks. Given the importance of the technique and its variants (like MCA), it is surprising that the authors devote such a limited room to the relevant framework. Consider that the discussion of CA in the context of two-mode networks seems not comprehensive as it misses the use of MCA in the same context - indeed, by merely citing it (one time) without discussing it properly. Both CA and MCA lead to appreciate visually (but also via the set of measures available in each tool) the similarities/differences between nodes and group of nodes in one mode by analysing their relational patterns with respect to the other mode.
Further, as the introduction states, there is an attempt at seemingly diminishing the power of visualization as a key feature of CA, as if it were a secondary concern in this domain. I think the authors should first consider carefully the actual utility of data visualization via CA/MCA and then also mind that, although their work aims to move ``beyond the focus on data visualization" (p. 2), they in fact make use of visual representation quite heavily in this paper. This seems to me rather misleading.

As regards CA and its ``conventional" - so to speak - interpretation which the authors criticize and want to go beyond, I disagree on the idea that using CA in a way different than the authors' would be less useful/informative or, even worse, ``off the mark". Here I specifically refer to a passage on p. 12:
"This interpretation implies (for instance) that two people who attend many of the same events or two groups with many members [in common?] will appear close in the plot. But this (still common) interpretation is off the mark. What the CA correspondence plot distance captures is, instead, people and groups that are surprisingly similar (e.g., from the point of view of a suitable null model, like independence) after taking people's activity levels and the sizes of the groups they belong to into account. Thus, people who share memberships in small groups will be closer in the diagram than people who share memberships in big groups".
While I would agree that one can interpret the distances between units - and, even more aptly, between the units and the axes origin - on a CA plot in terms of distances from an independence condition, I would not dismiss the importance of evaluating, in the above fashion, how ``people who share memberships in small groups will be closer in the diagram than people who share memberships in big groups". This has clear substantive value for interpretation of the data thanks to use of CA.

As for the paper's argument, one may ask what's new in the current proposal and what's the point in it, beyond already known things. This issue appears to affect, for example, the opportunity to analyse the forms of structural or regular equivalence in two-mode networks. This theme is paid very little attention, though. And yet, it is very relevant to the paper - as the authors seem at least to know, as they mention it (one time only, on p. 14). In the course of the paper there are some passages where the point would be relevant, as is the case with the bottom of p. 10, when the authors recall the works of Doreian, Batagelj and Ferligoj (2004), Kovács (2010) and Lizardo (2024), and assert that their own results are comparable to the latter's: "This indicates that persons and groups receive similar scores in the first dimension of CA only when they have similar connectivity similar to similar groups, where group similarity is defined dually as having members in common who are themselves similar." This is clearly a matter of positional equivalence, which deserves more room in the paper, in my opinion.

In addition, I would like to stress that the article sets forth an argument that, in the beginning, revolves around centrality, but ends with a more detailed discussion about how to assess relational similarity in two-mode networks. Here, there seems to be a lack of consideration regarding other relevant approaches. Blockmodeling, for instance, is not duly considered, although it is well known as a sort of gold standard in the analysis of equivalences - particularly generalized equivalences (I suggest seeing the book by Doreian, Batagelj, and Ferligoj, 2005, in addition to the Social Networks article by the same authors that is cited in the paper).

Consider two paragraphs on p. 14. Above, it is highlighted that CA, as it is shown in the plot of Figure 5(c), namely in the authors' usage, can show that some ``core events […] are more inclusive of peripheral members" while others ``are more "cliquish," including only core members". Note that this way of assessing the partitions is akin to that provided by blockmodeling for two-mode networks when dealing with core-periphery structures. On the same p. 14, below (in section 4.6), when GRS and the ``SimRank" measure are considered, the theme of positional equivalences (mainly regular equivalence) is chiefly relevant (but not mentioned) to the assertions ``People are similar if they belong to similar groups" and ``Groups are similar if they share similar members". I would like the authors to pay more attention to this absence in their discussion. More explicitly, what does the authors' proposal add to the analysis of structural/regular/generalized equivalences as it is already treated via blockmodeling (and also via MCA)?

I also think about other analytic strategies like Cluster Correspondence Analysis (Cluster CA) applied to two-mode networks, for which I may suggest reading and referencing relevant works (such as D'Ambrosio, Serino and Ragozini, 2021).

As a more general comment, I think that the paper looks like a thorough discussion of different ways of calculating and representing recursive centrality measures for two-mode networks. This is fine, unless one considers that the attempt turns out to be a sort of juxtaposition of assessments of each approach to centrality and position in two-mode networks.

The problem is that the recurring comparison between the different approaches, albeit interesting, is poorly leveraged to sustain the article's argument. It seems that by considering each step of that process of comparing different approaches one misses the whole picture. (Consider also the fact that the discussion/conclusion section is still not sufficiently elaborate, and this does not help in this regard.)

In sum, my contention is that the authors' argument should be strengthened to show what worth is reading this piece beyond already available literature. I think the paper is instructive as it discusses analytically the relevant topics, but it should be clearer about how this work will have an impact on the debate regarding CA, two-mode networks and the search for ways of better assessing the relational patterns in such networks.

Other minor points are as follows:

- The Highglights section should be improved in the sense that it looks now like a sort of abstract, while it needs to appear as a bullet-like summary of the paper that will make its argument better understood at the first glance

- Figure 3 is not clear, in particular it should have a proper layout where the content of the current insets would be shown separately (now it is a bit confusing)

\end{document}
\documentclass[]{letter}
\begin{document}
\textbf{Reviewer 1}

Given the proliferation of methods for ranking nodes in a network, along with the resultant confusion that confronts the analyst, there should be a special place in heaven for papers that show how seemingly different methods are related (and conversely). This paper resides in that special place. In economics a major paper published in PNAS by Hidalgo and Hausmann (2009), cited already over 3,700 times, develops and works with a measure of economic complexity - what they call the Method of Reflections. Their article (including its supplementary material) does not contain the words ``centrality," ``eigenvector," or ``correspondence analysis," and yet you (relying in part on your synthesis of a burgeoning multidisciplinary literature, and in part on your own novel insights) show how H\&H's reflective method is related to these other concepts, including how, in the limit, their method reproduces the first dimension of a correspondence analysis (CA), and therefore provides analysts with a ``new" way to understand what CA is doing, and how it works. A major theme in your paper is the similarities and (in particular) differences between Bonacich centrality scores (1991, Bonacich \& Lloyd 2001), a bedrock of network analysis, and correspondence analysis.

Among the insights in this paper that should be especially valuable to network analysts and that should affect research practice is that the CA plot captures, not that people who attend many of the same events will appear close in the plot (and conversely for similar groups), but rather that people who are surprisingly similar (from the point of view of a null model such as independence) will be close in the plot, after taking people's activity levels and sizes of the groups they belong to into account (and dually for groups; your p. 12). Another such insight concerns when (respectively) Bonacich centrality vs CA is most appropriate to be used.

Your paper provides an exploration of the concept of duality that is profound, in that the various methods that you bring together and relate to one another all are based on the idea that each "mode" of a two-mode network is constituted by the other mode. All the methods you relate are ways to model this co-constitution. Correspondence analysis and related techniques are core methods for the analysis of duality. In this sense, your paper is a very good fit indeed to this special issue.

In terms of its overall organization, your paper is (probably, necessarily) complicated. For me, the discussion / conclusion (Section 5) is much more clear and powerfully written than is the abstract and Introduction (Section 1). As you think about improving the paper, I hope you will find ways to assert the main Section 5 points crisply at the beginning of the paper. (For example, I would like to see a "Highlights" bullet point that says that Bonacich centrality is useful for revealing the position of the nodes in a center-periphery pattern, to the extent one exists, whereas CA is useful for revealing the presence of different groups or communities in the data, to the extent that the dual network is not homogeneous.)

Now I will turn to more specific comments. Importantly, there are 4 errors in the matrix ("Southern Women data") that you analyze. For this dataset, you (in the note to Table 1) cite Doreian et al (2004, Table 4). Among the attendees at Event E4, you show Frances, Eleanor, and Ruth. However they did not attend this event. You also show Nora as absent from Event E8; however, Nora attended E8. These four errors are discrepant in comparison to your cited source (Doreian et al, 2004, Table 4), and they are discrepant as well in comparison to the original source (Davis, Gardener, and Gardener, 1941, p. 141, which has events in the same order as in your Table 1, and a slightly different order of women; this table is reproduced in Freeman, 2003, p. 42). The data table in Kovacs (2010, p. 205) is correct (it is the same table as in Doreian et al), whereas the data table in Lizardo (2024, given in the qmd file in the supplementary material published for that article, lines 22-48 of the script) has the same 4 errors as in your manuscript.

So you will need to correct your Table 1 and, more importantly, all the subsequent analyses. In preparing this review, I have reproduced many of your analyses on the correct data, and it seems to me that, for the most part, the differences are not substantial and not substantive. Nonetheless, using the correct data is of course a requirement.

I have some resistance to your term "reflective centrality" (Section 2 title) to characterize the scores of the respective modes - precisely because, as you importantly show, in the limit the reflective scores are first dimensions of a CA, and (as you also importantly emphasize) the first CA dimension is about finding groups, not about ordering nodes. Hidalgo and Hausmann (2009) do not use the term "centrality" for the reflective scores they define and compute. I wonder whether "reflective scores" might be a better term; I'm not sure, and simply ask you to think about it.

What complicates things, as you know, is the difference between odd and even iterations of the reflective algorithm (see your comments in the first paragraph on p. 4, as well as comments of H \& H). One point you might consider adding is that, for the first even iterations (iterations 2, 4, 6, 8, and 10), the magnitude (absolute value) of the correlation of reflected scores with Bonacich centrality (for persons, and for groups) tends to increase with subsequent iterations (and, in the limit, is .682 for persons by iteration 40 for persons), whereas, for the first odd iterations (iterations 3, 5, 7, 9, 11), the magnitude of the correlation decreases (toward a limiting value, once again, of .682 by iteration 41). Can you relate this to the differences between odd and even iterations that you (and H\&H) discuss?

One of the high points of your paper - bringing great clarity, and of great usefulness to practitioners - is your Fig. 2, showing how the first (person and group) dimensions of the CA reveal ``groupi-ness" (my poor word choice), whereas the dual Bonacich centrality scores reveal the extent to which a center-periphery pattern obtains in the data. A point you miss here (however, I apologize if you do say this somewhere) is that the second dimension of Bonacich centrality (that is, the second dimension of the SVD of an affiliation matrix like Table 1) looks very much like the first CA dimension, emphasizing communities rather than center-periphery. This comment, like that of the previous paragraph, I think helps you to show how highly entangled are the concepts of reflective scores and centrality, and (therefore) how important it is to recognize when each is appropriate.

(In Fig. 2, both parts (a) and (b), nothing substantive or mathematical would change if you keep the row ordering you show, but reverse the column ordering. However, reversing the column ordering would help the reader to see, in panel (a), the groups, and, in panel (b), the center and the periphery. A similar comment applies to several of your other figures.)

What confused me about eqs. (23) and (24) is that I don't think the (respective) expressions in parentheses are symmetric matrices, and so I don't see how to take an eigenvector. (Each of the rows, but not the columns, sum to 1.)
    
Figs 3, 4, 5, and 6 each have multiple panels (referred to as a, b, c, etc), but, in the current version of the paper, it is difficult or impossible to figure out which part of the figure has which label (a, b, c, …). Hopefully the formatting here can be improved.

Reference not cited in your paper:

Freeman, L.C. (2003). "Finding Social Groups: A Meta-Analysis of the Southern Women Data." Pp. 39-77 in Dynamic Social Network Modeling and Analysis: Workshop Summary and Papers, edd. R. Breiger, K. Carley, P. Pattison. Washington, DC: National Academies Press.

\end{document}

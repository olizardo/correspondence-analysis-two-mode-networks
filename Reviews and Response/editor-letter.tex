\documentclass[]{letter}
\begin{document}

Dear Author,

Thank you for submitting your manuscript to the special issue of Social Networks on “Duality.” To assist with its evaluation, we enlisted two reviewers with significant expertise and a direct interest in the focal topic of your paper. We find their comments highly relevant and constructive. We have also read your paper independently, approaching it from the perspective of a general reader of Social Networks who may not have direct experience in your specific area of interest but would be drawn to innovative work that advances our understanding of network phenomena and – more specifically, network measurement and modeling.

Overall, we believe that incorporating the insightful feedback provided by the reviewers will significantly strengthen your manuscript and make it more compelling. Therefore, we are pleased to invite you to revise and resubmit your work for further consideration. Please note, however, that the opportunity to revise your manuscript does not guarantee its inclusion in the special issue.

We generally agree with the reviewers’ comments and strongly encourage you to take all of them into account, with a special focus on the key points I highlight below.

Reviewer 1 (R1) clearly appreciates the goals of your work and endorses its overall objectives. However, R1 raises concerns about potential errors in the matrix you analyze (specifically, the “Southern women data”). Since this is a canonical dataset, it is crucial that the version you use aligns with those employed in the studies you reference as benchmarks. If there are any discrepancies, please explain how and why the data you used differ. R1 offers some helpful advice on how to address this.

R1 also points out the need for further discussion on the odd and even iterations of the “reflective algorithm.” More generally, R1 finds that your "Discussion and Conclusions" section is much stronger and clearer than the Introduction. We share this view and encourage you to revisit your Introduction, clarifying the paper’s contribution and highlighting its novelty compared to existing work.

Reviewer 2 (R2) provides similar feedback, emphasizing the need for greater clarity around the distinct contribution of your research and its implications for future studies of social networks and beyond. To summarize R2's key point: What will readers learn from your paper that they did not know before? And what will they be able to do after reading your paper that they could not do before? I believe you can address these questions directly and convincingly. Both reviewers encourage you to make these points more explicit, especially in the Introduction and Motivation sections.

R2 also suggests broadening the context of your work by acknowledging additional relevant research on two-mode networks (2MN). For instance, consider reviewing the dual projection approach proposed by Everett and Borgatti in their 2013 paper published in a special issue of Social Networks on 2MN, as well as the hierarchy of dependence relations discussed by Robins and Pattison in the same special issue. Additionally, the more recent work of Lerner and Lomi (2022, Journal of Complex Networks), which uses the same “Southern Women” data, offers a framework that resonates with your own – albeit in a different modeling framework.

Finally, we agree with R2 that the link between your work and Breiger’s duality principle remains somewhat underdeveloped in the current manuscript. Strengthening this connection would greatly enhance the theoretical depth of your paper and align it more convincingly with the theme of the special issue.

We would like to bring to your attention a limited number of notational issues that may benefit from some clarification in the revised draft.

- p. 4 of "the sum of the q-1 centralities" is a little confusing. If we correctly understand your point, q-1 refers to the iteration rather than how many centralities are being summed. Perhaps this could be that could be reworded.
- Notation in eq. 19 and 20 seems slightly awkward - the left-hand side should somehow show that those are the transformed values, not the same as on the right-hand side (which currently is written the same way).
- For eq. 21 and 22, it might be helpful to note how the eigenanalysis of the two product matrices relates to the SVD of the original.

In conclusion, we believe the reviewers have provided several valuable and actionable recommendations that will help bring your manuscript to its next stage of development. We hope you agree that addressing their feedback will ultimately enhance the quality and impact of your work.

Thank you again for considering Social Networks as a potential outlet for your research. We look forward to reviewing your revised submission.

Sincerely,

Alessandro Lomi and Philippa Pattison
Executive Guest Editors, Social Networks
\end{document}


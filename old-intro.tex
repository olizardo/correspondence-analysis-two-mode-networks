This paper contributes to the stream of previous work applying CA to analyze two-mode data. Nevertheless, it departs from previous efforts to highlight aspects of CA for two-mode data analysis in SNA that have not been emphasized or treated in detail before. Particularly, I clarify the similarities and differences between the dual scores computed by CA on both modes and those obtained from the eigenvector-based approach proposed by \citet{bonacich1991simultaneous} for obtaining dual centralities in two-mode networks. These linkages are made clearer by recasting both CA and the Bonacich centralities as variations of a ``reflective'' strategy for analyzing two-mode network data via iterative dual scoring \citep{yang2022comparing, yang2020birank}, which proceeds by assigning substantively meaningful scores to the nodes in each set of a two-mode network using an iterative procedure equivalent to a spectral decomposition of a proximity matrix based on the one-mode projection. In this vein, I show the CA approach to two-mode network analysis---re-invented in some corners of Network Science under the guise of the ``economic complexity index'' \citep{hidalgo2009building, mealy2019interpreting, van2021correspondence}--- is part of a more general toolkit for the analysis of two-mode networks based on the ``dual projection'' approach \citep{everett2013dual}.

While the ``economic complexity'' interpretation of the CA scores on the first dimension has treated the resulting scores as ordinal rank measures---in the style of a centrality index---I synthesize recent work that shows that they are better seen as partitioning the two sets of nodes into \textit{similarity classes} based on their patterns of connectivity to the \textit{most similar} nodes in the other set \citep{kovacs2010generalized, lizardo2024two}. Thus, rather than ordering nodes on a unidimensional axis based on connectivity---like Bonacich's \citeyearpar{bonacich1991simultaneous} eigenvector-based approach---the first CA dimension extracts latent \textit{positional} information on people and groups. Particularly, there is a suggestive similarity between the scores obtained from the first dimension of the CA of the two-mode affiliation matrix and recent work on ``generalized relational similarity'' in two-mode networks \citep{kovacs2010generalized, lizardo2024two}. 

Of course, insofar as the Bonacich dual centrality approach is based on a spectral decomposition approach to two-mode network analysis (either via SVD of the rectangular affiliation matrix or eigendecomposition of the projection matrices), it can also be thought of as extracting latent positional information on both modes, especially when higher order axes are considered. The question then becomes, what is the difference between the positional information extracted by CA versus the dual centrality approach. I show that 

CA recovers clusters of entities (e.g., people) linked by their similar connectivity patterns to \textit{similar} entities in the other mode (e.g., groups), thus following a clustering logic of ``regular equivalence.'' In this way, CA can be used as an approach to indirect blockmodeling in two-mode networks. This is similar to the way CA has been developed and used in some disciplines (e.g., computer and information science), namely, as a \textit{spectral clustering} method for detecting most similar node subsets in bipartite graphs \citep{wu2022spectral}. Overall, CA emerges as a tool central to the usual social-network-analytic task of extracting underlying structure from two-mode (and multi-mode) relational data.
A final effort worth mentioning in this regard is an intriguing recent contribution by \citet{desposito2014use}, who convincingly advocate for using Multiple Correspondence Analysis (MCA) for the joint display of persons and groups in affiliation networks aimed at elucidating dual linkages between actors and events. The key difference between CA and MCA is that MCA works with a transformed version of the affiliation matrix, in which the rows are treated as ``cases'' and the columns as categorical attributes of the case, where each possible value represents a modality that the column object can take with respect to the cases. In the case of binary network data, the affiliation matrix is transformed into a new matrix $\mathbf{Z}$ with the same number of rows (persons) as the original and with twice the number of columns representing either membership ($z_{pg+} = 1$) or non-membership ($z_{pg-} = 1$) in that a group (disjunctive coding). The MCA scores are obtained via SVD of a centered and doubly normalized version of $\mathbf{Z}$. \citet{desposito2014use} discuss various advantages of the MCA graphical representation of two-mode networks, including the capacity to produce interpretable joint displays of actors and events and the ability to incorporate exogenous actor and event attributes as supplementary variables. 

In a related paper, more directly germane to the discussion that follows, \citet{desposito2014comparison} compare CA and MCA as tools for representing the structural similarity between nodes in a two-mode network (actors in their case). Specifically, they analyze the significance of the typical chi-square distances defined from the scores obtained from the simple CA of the affiliation matrix as compared to the MCA of the indicator matrix of persons by groups with events treated as disjunctively coded binary variables measured over the actors (as just described). The focus of the analysis is the extent to which the $\chi^2$ distance of actor profiles obtained from the affiliation matrix $\mathbf{A}$ or the disjunctive coded matrix $\mathbf{Z}$ best approaches structural equivalence. That is, \citet{desposito2014comparison} seek to ascertain whether CA or MCA is the best analytic tool to identify structurally equivalent actors in a two-mode network.  

Interestingly, \citet[115]{desposito2014comparison} note that by just examining the respective formulas defining the profile distances in CA versus MCA, it can be seen the CA is \textit{not} a good tool for revealing clusters of structurally equivalent actors. The reason, is that the CA profile distance ``adopts a peculiar and more complex weighting system that could lead to a distorted representation of the actual relational patterns" because the CA ``distance between actors (events) depends, not only on the pattern of participation (attendance), but also on the actor degree" and the event size. In addition, they note that in CA, small size events...are associated to larger weights...[and therefore] differences...related to such small size events have [a] stronger impact on the overall...distance than differences related to larger size events." \citet{desposito2014comparison} also note that, in contrast, the MCA definition of inter-profile distance between actors based on $\mathbf{Z}$ is not affected by the degree-weighting aspect, and that in this respect, ``[I]t turns out that MCA [distances] closely resembles the Euclidean distance." A conclusion that is supported by computational experiments conducted later in the paper, and which lead the authors to conclude that the MCA defined  $\chi^2$ distances ``better reproduces the \textit{actual} relational pattern embedded in affiliation networks'' \citep[122, italics added]{desposito2014comparison}.
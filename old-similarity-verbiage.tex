To further appreciate the links between CA, HH reflective scoring, and community partitioning in two-mode networks, note that $S_p = \mathbf{AD}g^{-1}\mathbf{A}^T$ in~\ref{eq:dam1} can be thought of as giving the similarity between pairs of people weighted by the size of the groups they belong to (so that people are more similar when they share smaller memberships).\footnote{Note that \citet[eq.2]{newman2001scientific} proposes a slightly modified version of this two-mode similarity score for people---in Newman's case, scientists---except that the one mode projection is normalized as $A(Dg - I)^{-1}A^T$; namely, the size of the ``group''---the number of co-authors in a paper---minus one. Substantively, this is unlikely to make much difference as the rank order of dyads by similarity between the two scores will be identical.} The same interpretation can be given to $S_g = \mathbf{A}^T\mathbf{D}p^{-1}\mathbf{A}$ in~\ref{eq:dam2}, which is the similarity of groups weighted by the activity of the people in them (so that groups are more similar when they share members who do not belong to many other groups). As \citet{van2021correspondence} show, the eigenvectors corresponding to (the Laplacian\footnote{For the similarity matrix $S$, the Laplacian is defined as $D-S$ where $D$ is the matrix containing the degrees of either people or groups in the diagonal and zeroes in every other cell.} of) these similarity matrices are equivalent to $C^R$. This means that the relative spread of the eigenvalues of the degree-weighted similarity matrix provides information concerning the quality of the resulting community partition. These are shown as inset plots in Figure~\ref{fig:ca-eigvec} for both people and groups. Note that in both cases, the first two eigenvalues separate from the rest, strongly indicating a dual-community structure in the Southern Women two-mode network. 

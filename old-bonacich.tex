This ambiguity in interpretation may stem from the fact that CA is usually seen as a technique to generate a plot that provides a ``low-dimensional approximation to the input data'' \citep[125]{faust2005using}, where the ``input data'' is presumed to be the original affiliation matrix $\mathbf{A}$. But as we have seen, this is \textit{not} what CA is designed to do. Instead, CA is meant to provide a low-dimensional approximation of a \textit{transformed} version of the input data, where the transformation is meant to adjust for people's activity levels and group sizes. Notably, if a low-dimensional representation of the original ``input data'' ($\mathbf{A}$) is what we are after, this may be more closely approximated by the first two eigenvectors of the usual one-mode projections of the matrix (where the first is, of course, the standard Bonacich dual centrality score). 

To illustrate, Figures~\ref{fig:bon-sim}(a) and~\ref{fig:bon-sim}(b) shows the ``raw'' (unweighted by other-mode degree) similarity scores for persons ($a_{pp'} = \sum_g a_{pg}a_{p'g}$) and groups ($a_{gg'} = \sum_p a_{pg}a_{pg'}$), with the rows and columns sorted by the first eigenvector of the matrix, which is the usual Bonacich centrality score. Both similarity matrices reproduce the triangular, core-periphery structure we observed in the re-ordered affiliation matrix in Figure~\ref{fig:ca-v-bon}(b).  Figure~\ref{fig:bon-sim}(c) plots $C^B$ on the x-axis against the second eigenvector of the unweighted similarity matrix---a relatively unusual but not substantively unmotivated practice \citep{iacobucci2017eigenvector}.\footnote{Nodes are colored to a four-cluster k-means solution using the first six eigenvectors of the respective similarity matrices.}  We can see that the plot of the first two eigenvectors does a good job of recovering the raw connectivity structure of the Southern Women affiliation network, partitioning the core persons and groups (on the upper-right) from the more peripheral ones (on the lower-left). 

Moreover, if all we are interested in is capturing a low-dimensional representation of which people have particular affinities for which events (regardless of people's activity levels or group sizes), then Figure~\ref{fig:bon-sim}(c) does a better job of that than the usual CA correspondence plot in Figure~\ref{fig:ca-sim}(c). For instance, $\left[Flora, Olivia\right]$ do have a special affinity for $\left[E11\right]$ and $\left[Nora, Katherine, Sylvia\right]$ for $\left[E10, E12\right]$. In the same way, core events like $\left[E8, E9, E10\right]$---shown on the lower half of the plot---are different from core events $\left[E3, E4, E5, E6\right]$---shown in the upper half. The former are more inclusive of peripheral members while the latter are more ``cliquish,'' including only core members. 

This ``preferential attachment'' \citep{barabasi1999emergence} of core people to core events and peripheral people to peripheral events seems to be captured by the second dimension, left over after considering each node's Bonacich eigenvector centrality. For instance, $\left[E1, E2\right]$, although as poorly attended as other peripheral events, tends to include core members and thus appear closer to other core actors in the upper half of the plot. Similarly, the main difference between $\left[Ruth\right]$ and $\left[Sylvia\right]$, despite their comparable $C^B$ scores, is that the former's (closer to the upper half of the plot) event profile is composed mainly of core events. In contrast, peripheral events dominate the latter's attendance profile (shown in the bottom half of the plot), accordingly, $\left[Ruth\right]$ and $\left[Sylvia\right]$ are assigned to distinct clusters by the k-means algorithm. 

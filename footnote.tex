\footnote{More generally, I will restrict my attention to work that uses CA to shed light on \textit{duality relations} between persons and groups in two-mode networks with special attention to the substantive significance, in terms of typical network metrics, of the scores assigned to nodes in each mode. Clearly, one can use CA for the multidimensional scaling one or two-mode data tables more generally when these are extracted from multimode relations between persons, organizations, or any other set of actors, sometimes containing weighted or valued ties \citep{ragozini2014correspondence}, along with attribute data for nodes in each mode. Of particular note in this regard is an early paper by \citet{wasserman1990correspondence}, which provided a wide-ranging discussion of the uses of CA and MCA for relational data (including two-mode data) and the links between CA and statistical models for both networks (e.g., Holland and Leinhardt's independent dyadic choice model) and contingency tables (e.g., canonical correlation and log-linear models; see also \citet{wasserman1989canonical}). In their paper, \citet{wasserman1990correspondence} mainly focused on providing a general introduction to CA as a method for analyzing relational data taking the form of valued multiway contingency tables (e.g., sender, receiver, by tie value). Notably, however, \citet[35]{wasserman1990correspondence} then noted that CA had potential for the analysis of duality relations in two-mode (e.g., actor-event) networks, remarking that ``[t]he dual scaling and reciprocal averaging interpretations of correspondence analysis make CA potentially useful for analyzing such data," but that other than an early unpublished attempt by Romney and Boyd, this largely remained ``a relatively unexplored application."}